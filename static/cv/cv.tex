\documentclass[11pt, a4paper]{cv_template}
\pagestyle{empty}

\begin{document}

\setstretch{1.5}

\begin{paracol}{3}

{\Huge Denis Krylov}

{\Large Senior Software Engeneer}

{\large C++}
\switchcolumn
\href{mailto:krylovdenism@gmail.com}{\Large \faAt \  krylovdenism@gmail.com} \par
\href{https://linkedin.com/in/d-krylov/}{\Large \faLinkedin \ d-krylov} \par
\href{https://greenwichmeantime.com/time/armenia/yerevan/}{\Large \faMapMarker \ Yerevan (GMT + 4)}

\switchcolumn

\href{https://github.com/d-krylov/}{\Large \faGithub \ d-krylov} \par
%\href{https://d-krylov.github.io/} {\Large \faHome \ d-krylov.github.io}
\end{paracol}

\vspace{10pt}

\hrule

\vspace{10pt}

\begin{paracol}{2}

{\Large \faUser \ SUMMARY}

{\large Experienced C++ engineer with over 10 years of expertise in systems programming, compiler and language runtime development. Worked at Samsung on  a neural network compiler and at Huawei on a language virtual machine and an AOT compilator. Background in computer graphics, computational geometry, and applied mathematics.}

\switchcolumn

{\Large \faUniversity \ EDUCATION}

{\large 2004 - 2009} \par
{\large Ivanovo State Power Engineering University} \par
{\large Specialist Degree in Industrial Electronics}

\vspace{5pt}

{\Large \faLanguage \ LANGUAGES}

{\large English - Professional working proficiency} \par
{\large Russian - Native proficiency}

\end{paracol}


\vspace{10pt}

{\Large \faTools \ SKILLS}

\vspace{5pt}

{\hspace{1em} \large \textbf{Languages}: C++ 23, C, Assembler x86-64, GLSL, Python, TeX} \par
{\hspace{1em} \large \textbf{GPU API}: Vulkan, OpenGL, CUDA} \par
{\hspace{1em} \large \textbf{Miscellaneous}: Computer graphics, PBR, Linear Algebra}

\vspace{10pt}

{\Large \faBriefcase \ EXPERIENCE}

\vspace{5pt}

{\hspace{1em} \large \faBuilding \quad 2022 - now \quad \textbf{Professional development break}} \par

\begin{addmargin}[3em]{0em}
Dedicated time to deepening expertise in computer graphics, computational geometry, and mathematics. Studying formal methods and theorem proving using Lean and Coq.  
Developing personal projects in rendering, geometry processing, and formal verification. Project descriptions are available on my \href{https://github.com/d-krylov/}{GitHub}.
\end{addmargin}

{\hspace{1em} \large \faBuilding \quad 2020 -- 2022 \quad \textbf{Huawei Russian Research Institute}}

\begin{addmargin}[3em]{0em}
Contributed to the development of the \href{https://gitee.com/openharmony-sig/arkcompiler_runtime_core}{Ark Compiler Runtime}, the core of the multi-platform runtime for OpenHarmony. It provides execution support for Ark bytecode and infrastructure for languages such as Java and JavaScript. Key contributions:
\begin{itemize}[noitemsep, topsep=0pt]
\item Enhanced the bytecode assembler for more efficient code generation and a user-friendly API.
\item Added support for Java language features in the bytecode converter.
\item Implemented JavaScript support in the bytecode interpreter and AOT compiler.
\item Developed tools for performance testing of the bytecode verifier and integrated them into CI pipelines.
\item Optimized calls from the interpreter to AOT-compiled functions.
\end{itemize}
\end{addmargin}

{\hspace{1em} \large \faBuilding \quad 2018 -- 2020 \quad \textbf{Samsung R\&D Institute Russia}} \par
\begin{addmargin}[3em]{0em}
Contributed to the development of \href{https://github.com/Samsung/ONE}{ONE}, a frontend compiler and IR framework designed to optimize neural networks for efficient execution on various hardware accelerators. It is part of Samsung’s AI compilation toolchain. Key contributions:
\begin{itemize}[noitemsep, topsep=0pt]
\item Developed a testing framework for the compiler.
\item Added support for TensorFlow operators in the IR generator and backend.
\end{itemize}
\end{addmargin}

{\hspace{1em} \large \faBuilding \quad 2015 -- 2018 \quad \textbf{MCST (Microprocessor Company)}}
\begin{addmargin}[3em]{0em}
Participated in a project focused on autonomous verification of an RTL processor model. Key responsibilities included:
\begin{itemize}[noitemsep, topsep=0pt]
  \item Writing SPARC assembler tests to verify the L2 cache RTL model.
  \item Creating SPARC assembler tests for verifying arithmetic instructions.
  \item Developing a test generator for verifying ALU arithmetic instructions in RTL.
\end{itemize}
\end{addmargin}

\end{document} 
